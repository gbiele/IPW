\begin{sidewaystable}
	\resizebox{\textwidth}{!}{
\begin{tabular}{rLlLlLlLlLLlL}
 \hline
 & \multicolumn{2}{c}{$IPW$} & 
 \multicolumn{2}{c}{$AR$} &
 \multicolumn{2}{c}{$\delta=AR-IPW$} &
 \multicolumn{3}{c}{$\delta/\sigma_{IPW}$ } &
 \multicolumn{3}{c}{$\delta/\mu_{IPW}$ }  \\
&\multicolumn{1}{c}{\footnotesize m} & \multicolumn{1}{c}{\footnotesize HDI} &
 \multicolumn{1}{c}{\footnotesize m} & \multicolumn{1}{c}{\footnotesize HDI} &
 \multicolumn{1}{c}{\footnotesize m} & \multicolumn{1}{c}{\footnotesize HDI} &  
 \multicolumn{1}{c}{\footnotesize m} & \multicolumn{2}{c}{\footnotesize HDI} {\footnotesize log(RR)} &
 \multicolumn{1}{c}{\footnotesize m} & \multicolumn{2}{c}{\footnotesize HDI} {\footnotesize log(RR)} \\
 \hline

\end{tabular}}
\caption{Means and 90\% HDIs of exposures outcome associations and standardized bias. $\sigma_{IPW}$ and $\mu_{IPW}$ are standard deviation and mean of the posterior distribution of the IPW regression coefficients. To obtain the log risk ratio $log(RR)$, we first calculated the posterior probability that the bias is within the ROPE, divided it by the opposite probability and took the logarithm of quotient. Hence, negative $log(RR)$ indicate presence of bias. For example, a log(RR) of -1.6 means that the bias estimate is five times as likely to lie outside the ROPE.} 
\label{table:estimates}
\end{sidewaystable}
